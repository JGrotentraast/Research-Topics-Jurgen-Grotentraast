% This must be in the first 5 lines to tell arXiv to use pdfLaTeX, which is strongly recommended.
\pdfoutput=1
% In particular, the hyperref package requires pdfLaTeX in order to break URLs across lines.

\documentclass[11pt]{article}

% Remove the "review" option to generate the final version.
\usepackage[review]{ACL2023}

% Standard package includes
\usepackage{times}
\usepackage{latexsym}

% For proper rendering and hyphenation of words containing Latin characters (including in bib files)
\usepackage[T1]{fontenc}
% For Vietnamese characters
% \usepackage[T5]{fontenc}
% See https://www.latex-project.org/help/documentation/encguide.pdf for other character sets

% This assumes your files are encoded as UTF8
\usepackage[utf8]{inputenc}

% This is not strictly necessary, and may be commented out.
% However, it will improve the layout of the manuscript,
% and will typically save some space.
\usepackage{microtype}

% This is also not strictly necessary, and may be commented out.
% However, it will improve the aesthetics of text in
% the typewriter font.
\usepackage{inconsolata}
\usepackage{tabularx}
\usepackage{float}
\usepackage{listings}
\usepackage{pgfgantt}


% If the title and author information does not fit in the area allocated, uncomment the following
%
%\setlength\titlebox{<dim>}
%
% and set <dim> to something 5cm or larger.

\title{Decision support data warehouse for Dutch financial institutions for
process improvement and reporting}

% Author information can be set in various styles:
% For several authors from the same institution:
% \author{Author 1 \and ... \and Author n \\
%         Address line \\ ... \\ Address line}
% if the names do not fit well on one line use
%         Author 1 \\ {\bf Author 2} \\ ... \\ {\bf Author n} \\
% For authors from different institutions:
% \author{Author 1 \\ Address line \\  ... \\ Address line
%         \And  ... \And
%         Author n \\ Address line \\ ... \\ Address line}
% To start a seperate ``row'' of authors use \AND, as in
% \author{Author 1 \\ Address line \\  ... \\ Address line
%         \AND
%         Author 2 \\ Address line \\ ... \\ Address line \And
%         Author 3 \\ Address line \\ ... \\ Address line}

\author{Jurgen Grotentraast \\
  Student Data Science \& technology\\
  University of Twente \\
  j.grotentraast@student.utwente.nl\\}

\begin{document}
{\makeatletter\acl@finalcopytrue
  \maketitle
}
% \begin{abstract}
% Abstract
% \end{abstract}

\section{Introduction}
With the still-growing value of data in today's world, many organizations have invested in the development of a data warehouse. A data warehouse is used to store data differently to efficiently analyze business data \cite{gupta1997selection}. Data warehouses can be used for analyzing and improving business processes \cite{shahzad2009goal}, but also to get a better understanding of for example the financial situation of an organization \cite{lapura2018development}. While data warehouses are becoming popular in every sector, one sector in which data warehouses can be useful is the financial sector. Every organization wants to be financially healthy and a data warehouse can give insight into this.

One company that has been involved in the financial sector for a long time now, is Topicus.Finance. The focus within Topicus.Finance lies on three main sectors pension and wealth \cite{pension},  mortgages \cite{mortgages}, and lending \cite{businesslending}. Within each of these sectors, Topicus.Finance has multiple software applications for individuals or companies. In this study, one of these software applications for the lending side will be used as the basis for a case study. The application is called Fyndoo \cite{fyndoo}, and is used by almost all major banks and many smaller banks and other financial institutions that offer lending services in the Netherlands. Fyndoo is a software application that streamlines the lending application process for both the financial institution and the applicant. \\

In the Netherlands, every financial institution has to report to "De Nederlandsche Bank" (DNB), which in turn reports to the European Central Bank (ECB). While Fyndoo makes this reporting easier for clients of Topicus.Finance, the information that is currently available for clients is based on what Topicus .Finance believes is needed. Therefore, the clients of Topicus .Finance would like more insight into the reports they have to deliver to the DNB and ECB. Furthermore, while Fyndoo helps streamline the lending application process, clients of Topicus .Finance want more insight into the lending application process for themselves to see how this process can be improved outside of the software application. Topicus .Finance only develops generic functionalities, meaning that a feature should be useful to all clients before being implemented. In other words, they do not tailor customization for one specific client, only the configuration of the software. Therefore, the needs of all clients should be identified such that a new release of Fyndoo is a good solution for all clients. \\

There is plenty of research already conducted on the topic of data warehouse and process improvement. While some of the studies mentioned in the related works in section \ref{related} are quite useful for the study proposed in this report, for example, the studies comparing open-source tooling \cite{thomsen2009survey}, there is an overall lack of studies on what the most suitable approach and tooling for different domains is. While there is overlap in what is important in different domains, there are also many differences. Studies highlighting data warehousing in an educational setting \cite{lapura2018development, nejres2015analysis} might have applications in the financial domain, but there is no real evidence to support this. 

\subsection{Problem statement}
While major banks like Rabobank and ING already have their own data warehouses in place for process improvement and reporting to the DNB, smaller clients of Topicus .Finance do not have the resources to develop this themselves. The existing features of Fyndoo are unfortunately not enough, and the exact needs of the clients of Topicus .Finance are still unknown. Current research shows a lot of use for data warehousing in this regard, however, no focus has been given to Dutch processes, which differ from processes in other countries and therefore might require a different approach for decision support meaning research concentrating on other countries might be not directly applicable. Furthermore, most data warehouses focus only on process improvement or reporting, not both. Moreover, Topicus .Finance prefers to use open-source software, however, most research focuses on Data Warehousing with commercial tooling. The studies conducted on open-source software provide overviews of available options but do not showcase the most suitable tooling and approach to use for process improvement or reporting.

% \subsection{Research questions}
% The above problem statement leads to the following research question: 
% \textit{What are the possibilities of a decision support data warehouse for Dutch financial institutions for process improvement and reporting?}

% By answering the following sub-questions, the main research question can be answered.
% \begin{enumerate}
%   \item What are the reporting needs of Dutch financial institutions?
%   \item What are the processes that Dutch financial institutions want to monitor?
%   \item What data do these processes rely on?
%   \item What are the best practices for a decision support data warehouse?
%   \begin{enumerate}
%       \item What tools are most suitable for a decision support data warehouse?
%   \end{enumerate}
%   \item How useful is the proposed system? \label{rq:usefulness}
% \end{enumerate}

\subsection{Research questions}
The above problem statement leads to the following research question: 
\textit{What are the most suitable open-source Data Warehouse solutions and Data Warehouse approaches for Dutch financial institutions for process improvement and reporting?}

By answering the following sub-questions, the main research question can be answered.
\begin{enumerate}
  \item What are the reporting needs of Dutch financial institutions?
  \item What are the processes that Dutch financial institutions want to monitor?
  \item What are the key performance indicators (KPIs) of these processes?
  \item How do different open-source Data Warehouse solutions compare to each other?
  \item How do different Data Warehouse approaches compare to each other?
  \item How useful is the proposed system? \label{rq:usefulness}
\end{enumerate}

\section{Related work}
\label{related}
The following paragraphs will discuss related work on data warehousing for financial data; business process improvement in general; data warehousing for process improvement; data warehouse tooling; and data warehouse usefulness. The focus will be on what these research papers accomplished and the gap in the literature that the proposed research in this report can fill. \\

First of all, there are several studies on data warehousing specifically for financial data \cite{amertha2020data, lapura2018development}. These studies focus on the data that is produced by banks or the financial status of an organization rather than the processes that could be improved. Furthermore, while the paper of Amertha et al. \cite{amertha2020data} shows potential for reporting to management or another institution to which a bank needs to report, the research only shows the capabilities of data warehousing not which tooling and practices are used. Furthermore, the paper of Lapura et al. researches the potential for a data warehouse for the financial situation of a university \cite{lapura2018development}. While this does show the potential for decision support with a data warehouse, the research itself mostly showed the advantages of a data warehouse. It was again not focused on process improvement or the tooling used. \\

Second of all, there are several studies done on process improvement, and also more specifically in a financial context. These studies involve literature reviews into business process improvement methodologies \cite{zellner2011structured}, which found that existing approaches to business processes were often a black box and therefore not reproducible as a systematically structured approach. However, this study was conducted in 2011, new research has been done in the use of a data warehouse for process improvement. Other studies have looked into process improvement in the financial context. For example, one study developed a theory model for process improvement which provides an understanding of the outcomes of business process improvement initiatives that were researched and shows potential to extend upon existing solutions \cite{buavaraporn2013business}. While these studies show that process improvement is important and that data warehousing is a good solution for this, they do not offer the specifics of a data warehousing solution in terms of tooling and approach. \\

Third of all, as mentioned above, there has been a rise in studies conducted on the topic of data warehousing regarding process improvement. One study showed the potential of a decision support data warehouse in a case study on the Swedish healthcare sector \cite{shahzad2009goal}. While this study was a success, the healthcare industry is vastly different from the financial industry. The processes involved in financial institutions and the clients of Topicus .Finance might not be comparable. Furthermore, this research does not combine this process warehouse with reporting needs which financial institutions are obliged to share with the Dutch National Bank, and does not highlight the tooling or approach.\\

Fourth of all, studies on different data warehouse tooling and approaches were found. One study focussed on the comparison of different Extract, Transform, and Load (ETL) tooling \cite{etl_comparison}. However, none of these tools were open-source and the result was only an overview of these different tooling, it did not showcase the use of these tools in a case study. The study of Nejres showcased that open-source tooling nowadays is just as capable as commercial software \cite{nejres2015analysis}. While the study successfully showed that open-source tooling is just as good as commercial, it did not highlight which tooling or approach was used and the case study was done in the educational domain not the financial. The paper of Pulla et al. describes an extensive comparative study on open-source data quality tools \cite{pulla2016open}. Yet again, the study does not involve other tooling or a comparison of approaches most suitable for different domains and applications. Next, the study of Dymore et al. shows a performance analysis of a real-time data warehouse solution \cite{dymora2023performance}. The study mentions the use of open-source tooling, for example, Apache Hadoop, Apache Hive, Apache Druid, and Apache Kafka. Furthermore, the study does explain different approaches that can be used for designing a data warehouse. However, the focus of this study is on the performance of the Apache software in a real-time setting, not on the suitability of the approach. Finally, Thomsen et al. conducted a survey study on different open-source business intelligence software \cite{thomsen2009survey}. They considered several ETL tools, database management systems (DBMSs), On‐Line Analytical Processing (OLAP) servers, and OLAP clients. The study was a revision of their previous study in 2005, however, the study was conducted in 2008 and therefore still rather outdated. Furthermore, the result of this study is again only an overview of different tools in different categories, there is no case study to see how suitable these tools are in a real-life setting or what approach is most suitable.

Last, there are many research papers on the effectiveness of data warehousing \cite{al2023empirical, rahman2022empirical, ramamurthy2008data}. While data warehouse effectiveness is not the main topic of the proposed research, it is very relevant research in regards to answering research question \ref{rq:usefulness}. \\

\section{Method}
\label{method}
The list of tasks as shown below outlines the basic steps that are to be taken to answer the proposed research questions. Section \ref{approach} will give a more detailed description of each task and how this task will be completed.

\begin{enumerate}
    \item Interview clients of Topicus .Finance for their reporting needs and processes
    \item Find common ground in answers
    \item Find best practices for data warehousing in literature
    \item Gather dummy data
    \item Design and implement solution
\end{enumerate}

\section{Approach}
\label{approach}
The first step is to interview the clients of Topicus .Finance who use this software and want to analyze their processes and improve their reporting. Since this product will be used daily, it is important to know what the clients of Topicus .Finance want and need. For the reporting needs it is also important to understand which stakeholders are involved as reporting to internal bookkeeping will have other requirements than the DNB or ECB. Furthermore, it is important to find the KPIs for all the relevant processes. Next to clients, the product manager will also be interviewed. Interviewing the product manager will be a great place to start gathering initial information on these clients and their processes as well as get an initial understanding of what is expected by the DNB and ECB. Interviewing the clients themselves will presumably confirm what was already found during the in-house interviews as well as expand upon that knowledge to create a clear overview of what these clients would like to see. \\

The result of the interviews will be a list of processes with their KPIs, and a list of what needs to be reported to who. As mentioned before, Topicus .Finance has a policy where customizations are not made for specific customers, instead, their products are as general as possible such that every feature that is released can be of use to all clients and only needs to be configured for them. Therefore, the results of the interviews should consist of processes and reporting needs that are relevant to all clients. A very specific request from one client that is not relevant to others will not be taken into account. \\

In the meantime, more information on the different approaches and tooling for creating a data warehouse needs to be found. This information will be based on the book "Data Warehouse Systems: Design and Implementation" by Alejandro Vaisman and Esteban Zimányi \cite{vaisman2014data}. This book highlights all the components and steps relevant to building a data warehouse. This book was written in 2022 therefore the information in this book will almost certainly still be relevant, however, to ensure this, corroborating literature has to be found. This step should result in an overview which can be used to decide upon the approach and tooling to be used in the last step.\\

After gathering all the information on the processes involved and the reporting needs for financial institutes, and the best approach and open-source tooling for decision support data warehousing have been made, the design and implementation of the solution can start. The design and implementation of the solution will at least involve designing data storage in a stil-to-be-determined format based on the approach that is used; designing and implementing an ETL process with a still-to-be-determined tool; filling the data storage with data; and finally, creating dashboards that can be used for the decision support when improving processes as well as give insights on how the organization is doing keeping in mind the required reports needed by the DNB and ECB. Depending on the approach that is taken there might be more steps to be done during this phase. Topicus .Finance has test data available which can be used for the implementation of the solution. \\

Lastly, the created solution will be evaluated with the involved clients to see if the created dashboards are useful or if they need more fine-tuning. While this step is very important, most feedback will be gathered during the design and implementation. This feedback will mostly come from the product manager and the employees of Topicus .Finance who are responsible for the product. If time allows it the feedback gathered from the clients will be incorporated in the solution.

\section{Planning}
Figure \ref{gantt_chart} shows the preliminary planning for this research. The first ten weeks will be used for completing all analysis components of this research. In these ten weeks, the interviews have to be completed and analyzed and the information on best practices of data warehouse needs to be gathered. Furthermore, decisions have to be made regarding the tooling to be used for the implementation of the data warehouse. Since the interviews will be dependent on the availability of the clients of Topicus .Finance it is preferable to have other tasks planned during this period to stay on schedule. The literature study on data warehouse approaches and tooling is not dependent on other people. Therefore, this study can be conducted in parallel to the interviews. Furthermore, the different nature of each of these tasks gives a balance in the type of work that is required to complete them. This can help with motivation and focus as performing the same task for a long time can be very exhausting and can lead to a lack of focus. The next seven weeks will be used to design and implement the data warehouse. The final four weeks are used to receive feedback on the developed solution, finish writing the report, and prepare for the presentation.

\begin{figure*}[t]

    \begin{center}
    \begin{ganttchart}[y unit title=0.4cm,
    y unit chart=0.5cm,
    vgrid,hgrid, 
    vrule/.style={very thick, green},
    vrule label font=\bfseries,
    title label anchor/.style={below=-1.6ex},
    title left shift=.05,
    title right shift=-.05,
    title height=1,
    progress label text={},
    bar/.append style={rounded corners=3pt},
    bar height=0.7,
    group right shift=0,
    group top shift=.6,
    group height=.3]{1}{21}
    %labels
    \gantttitle{Week}{21}\\
    \gantttitlelist{1,...,21}{1} \\\\
    %tasks
    \ganttbar[bar/.append style={fill=purple}]{Prepare and plan interviews}{1}{2} \\
    \ganttbar[bar/.append style={fill=violet}]{Conduct and analyse interviews}{3}{10} \\
    \ganttbar[bar/.append style={fill=cyan}]{Literature study on approaches and tooling}{1}{8}\\
    \ganttbar[bar/.append style={fill=yellow}]{Decide on tooling and approach}{8}{10} \\ 
    \ganttbar[bar/.append style={fill=blue}]{Design and implement data warehouse}{11}{17}\\ 
    \ganttbar[bar/.append style={fill=red}]{Evaluation of proposed solution}{18}{18} \\
    \ganttbar[bar/.append style={fill=pink}]{Presentation}{20}{21} \\
    \ganttbar[bar/.append style={fill=teal}]{Writing}{2}{21} \\
    \ganttvrule{Green light}{17}
    \end{ganttchart}
    \caption{Gantt chart of preliminary planning final project}
    \label{gantt_chart}
    \end{center}
\end{figure*}

% Entries for the entire Anthology, followed by custom entries
\bibliography{bibliography}
\bibliographystyle{acl_natbib}

\end{document}
